\documentclass[10pt]{amsart}
\usepackage[
  hmarginratio={1:1},     % equal left and right margins
  vmarginratio={1:1},     % equal top and bottom margins
  textwidth=435pt,        % new text width
  heightrounded,          % always useful
  %bindingcorrection=5mm,  % binding correction
]{geometry}
 \geometry{letterpaper}                   % ... or a4paper or a5paper or ... 
 %\geometry{landscape}                % Activate for for rotated page geometry
 %\usepackage[parfill]{parskip}    % Activate to begin paragraphs with an empty line rather than an indent
 \usepackage{graphicx}
 \usepackage{amssymb}
 \usepackage{epstopdf}
 \usepackage[T1]{fontenc}
%  \usepackage{relsize}
 \usepackage{listings}
%  \newcommand{\tab}{\hspace*{2em}}
%  \DeclareMathSizes{10}{10}{10}{10}
%  
%  \DeclareGraphicsRule{.tif}{png}{.png}{`convert #1 `dirname #1`/`basename #1 .tif`.png}

\newcommand{\R}{\mathbb{R}}
\newcommand{\set}[1]{\{ #1 \}}
\usepackage{amsthm}
\newtheorem{thm}{Theorem}
\newtheorem{defn}[thm]{Definition}
\newtheorem{lemma}[thm]{Lemma}
\newtheorem{conj}[thm]{Conjecture}
\newtheorem{open}[thm]{Open Problem}
\newtheorem{prop}[thm]{Proposition}

\title{CS6113 Notes on Quantitative Information Flow}
\author{Lecturer: Andrew Myers -- Scribes: Sam Hopkins, Jed Liu}
\date{November 12, 2013}                                           % Activate to display a given date or no date

\begin{document}
\maketitle

We read \cite{smith} and \cite{clarkson_etal}.

\section{Introduction}
First proposed by Denning and Gray in the early 1990s, the basic goal of a
theory of quantitative information flow is to measure the information leaked by
a program to an attacker by magnitude of the mutual information between some
true distribution on secret data and the attacker's best guess at the
distribution of that data. Early models of quantitative information flow
directly used the Shannon mutual information $I(X ; Y)$. Roughly, $I(X ; Y)$
measures the (expected) reduction in uncertainty about the value of random
variable $X$ given by learning the value of $Y$ (formal definitions follow).

Doing information flow quantitatively immediately raises the following
objections, among others:

\begin{itemize}
  \item Na\"ively, $I(X;Y)$ seems to value all secret data equally. However,
    different pieces of high-security data might be worth more than
    others---not all leaked bits are equal. This problem is beginning to be
    addressed in work of Scedrov, but is still at least a partially-open
    question.
  \item It's not at all clear how to analyze code to measure its information
    leakage quantitatively. There is a POPL '07 paper on this problem, as well
    as a body of work on sampling executions of randomized programs and
    programs with randomized input data, as well as a POPL '07 paper on this
    problem.
  \item Dealing with nondeterminism in the program may pose a problem to a
    theory of quantitative information flow. Orthogonally, using Shannon mutual
    information directly does not appear to result in a theory that allows
    compositional reasoning about program executions: if $k$ bits are leaked in
    the first execution of a program and $k'$ in the second, we would like our
    theory to say that together the executions leak $k + k'$ bits, but that
    need not be true if we work directly with mutual information. The work of
    Clarkson et al. in CSF '05 addresses both of these problems simultaneously
    \cite{clarkson_etal}.
  \item Geoffry Smith claims that Shannon entropy is the wrong measure, because
    it fails to accurately capture what he calls the \emph{vulnerability} of a
    program, which is the probability that attacker's first guess at the high
    data is correct. This quantity is captured by \emph{min-entropy}. The work
    is discussed further below.
\end{itemize}


\section{Information Theory -- Definitions}
\begin{defn} Let $X$ be a discrete random variable over a probability space $\mathcal X$. The
  Shannon entropy of $X$, written $H(X)$, is
  \[
    H(X) = -\sum_{x \in \mathcal X} p(x) \log p(x) = E_{x \sim X}[-\log p(x)].
  \]
  $H(X)$ measures ``surprisal'' the average amount that one should be surprised
  by seeing the value of $X$. The definition originates in coding theory from
  the 1940s.
\end{defn}
\begin{defn}
  The conditional entropy of $X$ given $Y$, written $H(X|Y)$, is $E_{y \sim Y} H(X|Y = y)$,
  where $H(X | Y = y)$ is the entropy of the conditional distribution $p(x | Y = y)$.
\end{defn}
\begin{defn}
  The mutual information $I(X ; Y)$ between variables $X,Y$ is the entropy
  reduction in $X$ given by learning $Y$. Equivalent definitions are
  \[
    I(X;Y) = H(X) + H(Y) - H(X,Y) = H(X) - H(X | Y) = H(Y) - H(Y | X).
  \]
  Mutual information is symmetric and nonnegative. Conditional mutual information is
  defined analogously to conditional entropy.
\end{defn}



\section{Geoffrey Smith: Vulnerability and Min-Entropy}

\section{Clarkson et al: Beliefs}

\section{Summary of Discussion}
The discussion seemed to split into three questions.
\begin{itemize}
  \item There seem to be lots of proposed models of quantitative information
    flow, all of which have nontrivial differences. Is there a taxonomy of such
    models? Is there a list of desiderata for them? I.e., is there a list of
    theorems that should be provable about a model of quantitative information
    flow in order for it to qualify as such? The answer seems to be: no, but
    that would be pretty nice. Maybe some theorem that relates the leakage
    measured by the model to some notion in formulated in the language of
    differential privacy?
  \item Is there a connection between these leakage measures and differential
    privacy? Given that much of the privacy community seems to have settled on
    differential privacy as the right notion, this seems like an important
    question to address. One paper which seems to address this issue is
    \cite{barthe2011}.
  \item There are often natural coding interpretations of information-theoretic
    quantities which are enlightening to think about when using an entropic
    measure in the wild. How can we interpret the use of relative entropy in 
    the Clarkson et al. paper? Since relative entropy measures efficiency losses
    when coding a message according to the wrong distribution, one proposal is
    to interpret it as follows: suppose we have an inside attacker who learns 
    the high-security output but believes it to be distributed differently than
    it really is, and suppose that this attacker must exfiltrate the data on some
    bounded communication channel. Then the relative entropy between the actual
    distribution of the high data and the attacker's guess at the distribution
    should somehow measure how successfully the attacker can exfiltrate the data
    on his bounded channel.
\end{itemize}


\bibliographystyle{plain}
\bibliography{bibliography}


\end{document}
